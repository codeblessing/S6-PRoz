\documentclass[12pt, a4paper, oneside]{article}

\usepackage{amsmath}
\usepackage[margin=0.8in]{geometry}

\title{Programowanie rozproszone}
\author{Jakub Kwiatkowski 145356\\Paweł Strzelczyk 145217}
\date{}


\begin{document}
\maketitle

\section*{Problem}
$ W $ winiarzy (oznaczonych dalej przez $ W_{i} $) produkuje każdy po $ X_{i} $ litrów wina.\\
$ S $ studentów (oznaczonych dalej przez $ S_{j} $) konsumuje każdy po $ Y_{j} $ litrów wina.\\
Aby przekazać wino, winiarz musi wynająć bezpieczne miejsce $ B_{k} $.\\
Winiarz nie rozpocznie produkcji wina, dopóki nie odda wszystkiego, co już wyprodukował.\\
\\
Przy założeniach:
$$ i \in \left\{1, 2, \ldots, W\right\} $$
$$ j \in \left\{1, 2, \ldots, S\right\} $$
$$ k \in \left\{1, 2, \ldots, N\right\} $$
$$ \neg (\forall_{i \in \{1, 2, \ldots, W\}} \exists_{j \in \left\{1, 2, \ldots, S\right\}} X_{i} = Y_{j}) $$

\section*{Proponowane rozwiązanie}

Aby rozwiązać podany problem musimy poczynić dodatkowe założenia:

\begin{itemize}
    \item $ \sum X_{i} = \sum Y_{j} $
    \item Winiarz nie musi oddać całej partii jednemu studentowi ($ X_{i} \geq Y_{j} $)
    \item Student nie musi zaspokoić wszystkich swoich potrzeb u jednego winiarza ($ Y_{j} \geq X_{i} $)
    \item Student może w razie potrzeby zaspokoić tylko część swojego zapotrzebowania, jednak pozostała część musi zostać zaspokojona tak szybko jak to tylko możliwe.
\end{itemize}

Założenie pierwsze zapobiega problemowi nadprodukcji. Jeśli założymy że proces jest ciągły i nieskończony, to przybiera ono formę:
\begin{equation*}
    \lim_{n \to \infty}{n \times \sum X_{i}} = \lim_{n \to \infty}{n \times \sum Y_{j}}
\end{equation*}
i jest pomijalne w praktyce.\\
\\
Założenia drugie i trzecie zapobiegają sytuacji, w której popyt i podaż sumarycznie się równoważą, ale niemożliwy jest przydział całościowy.\\
\\
Założenie czwarte jest potrzebne w przypadku procesu ciągłego i nieskończonego, pozwalając na złamanie założenia pierwszego w czasie jednej iteracji
zakładając, że zostanie ono skorygowane w czasie następnych iteracji (zbyt mała podaż w i-tej iteracji zostanie zrównoważona nadpodażą w  i + 1 iteracji)
\subsection*{Opis algorytmu}

\begin{enumerate}
    \item Winiarz $ W_{i} $ ,,produkuje'' $ X_{i} $ litrów wina i ubiega się o bezpieczne miejsce $ B_{i} $:
    \begin{enumerate}
        \item  $ W_{i} $ rozsyła do wszystkich winiarzy wiadomość \texttt{REQ} zawierającą swój zegar Lamporta $ L_{W_{i}} $ oraz żądane miejsce $ B_{i} $.
        \item Po otrzymaniu wiadomości \texttt{ACK} (lub \texttt{REQ} z wyższymi zegarami Lamporta niż własny) od wszystkich winiarzy,
        $ W_{i} $ wchodzi do sekcji krytycznej i rozsyła wszystkim wiadomość $ M_{W_{i}} $ (\texttt{INFO}) zawierającą informacje o $ B_{i} $ i $ X_{i} $.
        \item Jeśli $ W_{i} $ otrzyma wiadomość \texttt{REQ} z zegarem Lamporta niższym niż własny odsyła \texttt{ACK} i dalej oczekuje na pozostałe \texttt{ACK}.
        \item Po rozgłoszeniu wiadomości $ M_{W_{i}} $, $ W_{i} $ wychodzi z sekcji krytycznej i rozsyła \texttt{ACK} wszystkim oczekującym.
    \end{enumerate}
    \item $ W_{i} $ rozpoczyna następną produkcję dopiero po otrzymaniu \texttt{RELEASE} zawierającego $ B_{i} $.
    
    \item Student $ S_{j} $ określa swoje zapotrzebowanie na $ Y_{j} $ litrów wina i ubiega się o dostęp do $ B_{k} $:
    \begin{enumerate}
        \item $ S_{j} $ wysyła do wszystkich studentów wiadomość \texttt{REQ} zawierającą $ B_{k} $ oraz ilość wina jaką zamierza stamtąd pobrać.
        \item Po otrzymaniu wiadomości \texttt{ACK} (lub \texttt{REQ} z wyższymi zegarami Lamporta niż własny) od wszystkich studentów,
        $ S_{j} $ wchodzi do sekcji krytycznej.
        \item Jeśli $ S_{j} $ wyczerpie zasoby $ B_{k} $, to rozsyła winiarzom wiadomość \texttt{RELEASE} zawierającą $ B_{k} $.
        \item Po wyjściu z sekcji krytycznej $ S_{j} $ rozsyła \texttt{ACK} wszystkim oczekującym.
    \end{enumerate}
    \item $ S_{j} $ ubiega się o kolejne $ B_{k} $ doputy, dopóki nie wypełni zapotrzebowania $ Y_{j} $.
\end{enumerate}

Zegary Lamporta są inkrementowane w zwykły sposób, jednak nigdy nie w trakcie rozsyłania wiadomości (rozesłanie wiadomości do grupy odbiorców jest traktowane jako operacja atomowa).

\end{document}