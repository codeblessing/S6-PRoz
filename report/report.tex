\documentclass[12pt, a4paper, twoside]{article}

\usepackage{amsmath}

\begin{document}

\section*{Problem}

$ W $ winiarzy (oznaczonych dalej przez $ W_{i} $) produkuje każdy po $ X_{i} $ litrów wina.\\
$ S $ studentów (oznaczonych dalej przez $ S_{j} $) konsumuje każdy po $ Y_{j} $ litrów wina.\\
Aby przekazać wino, winiarz musi wynająć bezpieczne miejsce $ B_{k} $.\\
Winiarz nie rozpocznie produkcji wina, dopóki nie odda wszystkiego, co już wyprodukował.\\
\\
Przy założeniach:
$$ i \in \left\{1, 2, \ldots, W\right\} $$
$$ j \in \left\{1, 2, \ldots, S\right\} $$
$$ k \in \left\{1, 2, \ldots, N\right\} $$
$$ \neg (\forall_{i \in \{1, 2, \ldots, W\}} \exists_{j \in \left\{1, 2, \ldots, S\right\}} X_{i} = Y_{j}) $$

\section*{Proponowane rozwiązanie}

Aby rozwiązać podany problem musimy poczynić dodatkowe założenia:

\begin{itemize}
    \item $ \sum X_{i} = \sum Y_{j} $
    \item Winiarz nie musi oddać całej partii jednemu studentowi ($ X_{i} \geq Y_{j} $)
    \item Student nie musi zaspokoić wszystkich swoich potrzeb u jednego winiarza ($ Y_{j} \geq X_{i} $)
    \item Student może w razie potrzeby zaspokoić tylko część swojego zapotrzebowania, jednak pozostała część musi zostać zaspokojona tak szybko jak to tylko możliwe. 
\end{itemize}

Założenie pierwsze zapobiega problemowi nadprodukcji. Jeśli założymy że proces jest ciągły i nieskończony, to przybiera ono formę:
\begin{equation*}
    \lim_{n \to \infty}{n \times \sum X_{i}} = \lim_{n \to \infty}{n \times \sum Y_{j}}
\end{equation*}
i jest niespełnialne w praktyce.\\
\\
Założenia drugie i trzecie zapobiegają sytuacji, w której popyt i podaż sumarycznie się równoważą, ale niemożliwy jest przydział całościowy.\\
\\
Założenie czwarte jest potrzebne w przypadku procesu ciągłego i nieskończonego, pozwalając na złamanie założenia pierwszego w czasie jednej iteracji
zakładając, że zostanie ono skorygowane w czasie następnych iteracji (zbyt mała podaż w i-tej iteracji zostanie zrównoważona nadpodażą w  i + 1 iteracji)
\subsection*{Opis algorytmu}

\begin{enumerate}
    \item Winiarz $ W_{i} $ "produkuje" $ X_{i} $ litrów wina i ubiega się o bezpieczne miejsce $ B_{i} $.
    \item Po otrzymaniu $ B_{i} $ (wejście do sekcji krytycznej), $ W_{i} $ rozsyła informację o $ X_{i} $ do wszystkich studentów i wychodzi z sekcji krytycznej.
    \item Student $ S_{j} $ określa swoje zapotrzebowanie na $ Y_{j} $ litrów wina i ubiega się o dostęp do winiarzy.
    \item Jeśli dostępne są zasoby wina, $ S_{j} $ otrzymuje dostęp zgodnie z algorytmem przydziału, w przeciwnym przypadku oczekuje na wiadomość od $ W_{i} $.
    \item Po otrzymaniu dostępu do winiarzy (wejście do sekcji krytycznej) $ S_{j} $, wybiera winiarzy od których weźmie wino
    \item Po wyborze winiarzy $ S_{j} $ odbiera wino i (jeśli odebrał całą partię) zwalnia $ B_{i} $ i informuje  o tym $ W_{i} $.
    \item Jeśli $ S_{j} $ odebrał całe wino jakie było dostępne, to rozgłasza on wyczerpanie zasobów wina.
    \item $ S_{j} $ wychodzi z sekcji krytycznej.
    \item Jeśli zapotrzebowanie $ S_{j} $ nie zostało w całości zaspokojone, to ponownie ubiega się on o dostęp do winiarzy dostając najwyższy priorytet.
\end{enumerate}

\end{document}