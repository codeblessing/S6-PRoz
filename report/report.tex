\documentclass[12pt, a4paper, oneside]{article}

\usepackage{amsmath}
\usepackage{amsfonts}
\usepackage[margin=0.8in]{geometry}
\usepackage{alltt}

\title{Programowanie rozproszone}
\author{Jakub Kwiatkowski 145356\\Paweł Strzelczyk 145217}
\date{}


\begin{document}
\maketitle

\section*{Problem}
$ W $ winiarzy (oznaczonych dalej przez $ W_{i} $) produkuje każdy po $ X_{i} $ litrów wina.\\
$ S $ studentów (oznaczonych dalej przez $ S_{j} $) konsumuje każdy po $ Y_{j} $ litrów wina.\\
Aby przekazać wino, winiarz musi wynająć bezpieczne miejsce $ B_{k} $.\\
Winiarz nie rozpocznie produkcji wina, dopóki nie odda wszystkiego, co już wyprodukował.\\
\\
Przy założeniach:
$$ i \in \left\{1, 2, \ldots, W\right\} $$
$$ j \in \left\{1, 2, \ldots, S\right\} $$
$$ k \in \left\{1, 2, \ldots, N\right\} $$
$$ \neg (\forall_{i \in \{1, 2, \ldots, W\}} \exists_{j \in \left\{1, 2, \ldots, S\right\}} X_{i} = Y_{j}) $$

\section*{Proponowane rozwiązanie}

Aby rozwiązać podany problem musimy poczynić dodatkowe założenia:

\begin{itemize}
    \item $ \sum X_{i} = \sum Y_{j} $
    \item Winiarz nie musi oddać całej partii jednemu studentowi ($ X_{i} \geq Y_{j} $)
    \item Student nie musi zaspokoić wszystkich swoich potrzeb u jednego winiarza ($ Y_{j} \geq X_{i} $)
    \item Student może w razie potrzeby zaspokoić tylko część swojego zapotrzebowania, jednak pozostała część musi zostać zaspokojona tak szybko jak to tylko możliwe.
\end{itemize}

Założenie pierwsze zapobiega problemowi nadprodukcji. Jeśli założymy że proces jest ciągły i nieskończony, to przybiera ono formę:
\begin{equation*}
    \lim_{n \to \infty}{n \times \sum X_{i}} = \lim_{n \to \infty}{n \times \sum Y_{j}}
\end{equation*}
i jest pomijalne w praktyce.\\
\\
Założenia drugie i trzecie zapobiegają sytuacji, w której popyt i podaż sumarycznie się równoważą, ale niemożliwy jest przydział całościowy.\\
\\
Założenie czwarte jest potrzebne w przypadku procesu ciągłego i nieskończonego, pozwalając na złamanie założenia pierwszego w czasie jednej iteracji
zakładając, że zostanie ono skorygowane w czasie następnych iteracji (zbyt mała podaż w i-tej iteracji zostanie zrównoważona nadpodażą w  i + 1 iteracji)
\subsection*{Opis algorytmu}

\begin{enumerate}
    \item Winiarz $ W_{i} $ ,,produkuje'' $ X_{i} $ litrów wina i ubiega się o bezpieczne miejsce $ B_{i} $:
          \begin{enumerate}
              \item  $ W_{i} $ rozsyła do wszystkich winiarzy wiadomość \texttt{REQ} zawierającą swój zegar Lamporta $ L_{W_{i}} $ oraz żądane miejsce $ B_{i} $.
              \item Po otrzymaniu wiadomości \texttt{ACK} (lub \texttt{REQ} z wyższymi zegarami Lamporta niż własny) od wszystkich winiarzy,
                    $ W_{i} $ wchodzi do sekcji krytycznej i rozsyła wszystkim wiadomość $ M_{W_{i}} $ (\texttt{INFO}) zawierającą informacje o $ B_{i} $ i $ X_{i} $.
              \item Jeśli $ W_{i} $ otrzyma wiadomość \texttt{REQ} z zegarem Lamporta niższym niż własny odsyła \texttt{ACK} i dalej oczekuje na pozostałe \texttt{ACK}.
              \item Po rozgłoszeniu wiadomości $ M_{W_{i}} $, $ W_{i} $ wychodzi z sekcji krytycznej i rozsyła \texttt{ACK} wszystkim oczekującym.
          \end{enumerate}
    \item $ W_{i} $ rozpoczyna następną produkcję dopiero po otrzymaniu \texttt{RELEASE} zawierającego $ B_{i} $.

    \item Student $ S_{j} $ określa swoje zapotrzebowanie na $ Y_{j} $ litrów wina i ubiega się o dostęp do $ B_{k} $:
          \begin{enumerate}
              \item $ S_{j} $ wysyła do wszystkich studentów wiadomość \texttt{REQ} zawierającą $ B_{k} $ oraz ilość wina jaką zamierza stamtąd pobrać.
              \item Po otrzymaniu wiadomości \texttt{ACK} (lub \texttt{REQ} z wyższymi zegarami Lamporta niż własny) od wszystkich studentów,
                    $ S_{j} $ wchodzi do sekcji krytycznej.
              \item Jeśli $ S_{j} $ wyczerpie zasoby $ B_{k} $, to rozsyła winiarzom wiadomość \texttt{RELEASE} zawierającą $ B_{k} $.
              \item Po wyjściu z sekcji krytycznej $ S_{j} $ rozsyła \texttt{ACK} wszystkim oczekującym.
          \end{enumerate}
    \item $ S_{j} $ ubiega się o kolejne $ B_{k} $ doputy, dopóki nie wypełni zapotrzebowania $ Y_{j} $.
\end{enumerate}

Zegary Lamporta są inkrementowane w zwykły sposób, jednak nigdy nie w trakcie rozsyłania wiadomości (rozesłanie wiadomości do grupy odbiorców jest traktowane jako operacja atomowa).

\section*{Implementacja}

W naszej implementacji proponowanego algorytmu przyjęliśmy następującą strukturę wiadomości:

\begin{alltt}
Message \{
    type,
    timestamp,
    sender,
    payload \{
        safehouse\_index,
        wine\_volume,
        last\_timestamp
    \}
\}
\end{alltt}

Pole \texttt{payload} jest zawartością specyficzną dla typu wiadomości.
Poniżej przedstawiono typy wiadomości i pole \texttt{payload} specyficzne dla nich.

\begin{itemize}

\item \texttt{WINEMAKER\_REQUEST}:
\begin{alltt}
payload: \{
    safehouse\_index
\}
\end{alltt}

\item \texttt{WINEMAKER\_ACKNOWLEDGE}:
\begin{alltt}
payload: \{\}
\end{alltt}

\item \texttt{WINEMAKER\_BROADCAST}:
\begin{alltt}
payload: \{
    safehouse\_index,
    wine\_volume
\}
\end{alltt}

\item \texttt{STUDENT\_REQUEST}:
\begin{alltt}
payload: \{
    safehouse\_index,
    wine\_volume
\}
\end{alltt}

\item \texttt{STUDENT\_ACKNOWLEDGE}:
\begin{alltt}
payload: \{
    safehouse\_index,
    last\_timestamp
\}
\end{alltt}

\item\texttt{STUDENT\_BROADCAST}:
\begin{alltt}
payload: \{
    safehouse\_index
\}
\end{alltt}

\end{itemize}

Niech $ \mathbb{W} $ będzie uporządkowanym wektorem winiarzy oraz $ \mathbb{S} $ będzie uporządkowanym wektorem studentów. Ponadto, niech $ \mathbb{B} $ będzie wektorem bezpiecznych miejsc. Wtedy:

\begin{flalign*} & \forall{w \in \mathbb{W}}: &\\ \end{flalign*}
\begin{enumerate}
    \item Niech $ i $ będzie indeksem $ w $ w wektorze $ \mathbb{W} $
    \item Niech $ b = i\mod{\|\mathbb{B}\|} $ będzie przypisanym do $ w $ indeksem bezpiecznego miejsca w wektorze $ \mathbb{B} $
    \item\label{alg:wmk:start} $ w $ rozsyła do pozostałych winiarzy wiadomość \texttt{WINEMAKER\_REQUEST} z informacją o $ b $
    \item $ w $ ustawia swój licznik zgód na $ 0 $
    \item Dopóki licznik zgód jest mniejszy od $ \|\mathbb{W}\| - 1 $:
    \begin{enumerate}
        \item $ w $ oczekuje na wiadomość $ m $
        \item Jeśli $ m $ jest typu \texttt {WINEMAKER\_ACKNOWLEDGE}, $ w $ zwiększa licznik zgód o $ 1 $
        \item Jeśli $ m $ jest typu \texttt{WINEMAKER\_REQUEST}:
            \begin{enumerate}
                \item Jeśli zegar $ m $ jest niższy od zegara wysłanej wiadomości lub indeks bezpiecznego miejsca jest różny od $ b $, $ w $ wysyła wiadomość zwrotną \texttt{WINEMAKER\_ACKNOWLEDGE}
                \item Jeśli zegar $ m $ jest równy zegarowi wysłanej wiadomości oraz identyfikator $ w $ jest większy od identyfikatora nadawcy $ m $, $ w $ wysyła wiadomość zwrotną \linebreak \texttt{WINEMAKER\_ACKNOWLEDGE}
                \item Jeśli zegar $ m $ jest większy od zegara wysłanej wiadomości lub zegar $ m $ jest równy zegarowi wysłanej wiadomości oraz identyfikator $ w $ jest mniejszy od identyfikatora nadawcy $ m $, $ w $ dopisuje nadawcę $ m $ do kolejki oczekujących na zgodę i zwiększa licznik zgód o $ 1 $
            \end{enumerate}
        \end{enumerate}
    \item $ w $ wybiera losowo liczbę jednostek wina $ v $, i rozsyła ją do wszystkich studentów wiadomością \texttt{WINEMAKER\_BROADCAST}
    \item $ w $ oczekuje na wiadomość $ m $ typu \texttt{STUDENT\_BROADCAST} z indeksem bezpiecznego miejsca równym $ b $
    \item $ w $ rozyła wiadomość \texttt{WINEMAKER\_ACKNOWLEDGE} do winiarzy w kolejce oczekujących na zgodę
    \item Wróć do \ref{alg:wmk:start}
\end{enumerate}

// TODO:
\begin{flalign*} & \forall{s \in \mathbb{S}}: &\\ \end{flalign*}
\begin{enumerate}
    \item Niech $ i $ będzie indeksem $ w $ w wektorze $ \mathbb{W} $
    \item Niech $ b = i\mod{\|\mathbb{B}\|} $ będzie przypisanym do $ w $ indeksem bezpiecznego miejsca w wektorze $ \mathbb{B} $
    \item\label{alg:stu:start} $ w $ rozsyła do pozostałych winiarzy wiadomość \texttt{WINEMAKER\_REQUEST} z informacją o $ b $
    \item $ w $ ustawia swój licznik zgód na $ 0 $
    \item Dopóki licznik zgód jest mniejszy od $ \|\mathbb{W}\| - 1 $:
    \begin{enumerate}
        \item $ w $ oczekuje na wiadomość $ m $
        \item Jeśli $ m $ jest typu \texttt {WINEMAKER\_ACKNOWLEDGE}, $ w $ zwiększa licznik zgód o $ 1 $
        \item Jeśli $ m $ jest typu \texttt{WINEMAKER\_REQUEST}:
            \begin{enumerate}
                \item Jeśli zegar $ m $ jest niższy od zegara wysłanej wiadomości lub indeks bezpiecznego miejsca jest różny od $ b $, $ w $ wysyła wiadomość zwrotną \texttt{WINEMAKER\_ACKNOWLEDGE}
                \item Jeśli zegar $ m $ jest równy zegarowi wysłanej wiadomości oraz identyfikator $ w $ jest większy od identyfikatora nadawcy $ m $, $ w $ wysyła wiadomość zwrotną \linebreak \texttt{WINEMAKER\_ACKNOWLEDGE}
                \item Jeśli zegar $ m $ jest większy od zegara wysłanej wiadomości lub zegar $ m $ jest równy zegarowi wysłanej wiadomości oraz identyfikator $ w $ jest mniejszy od identyfikatora nadawcy $ m $, $ w $ dopisuje nadawcę $ m $ do kolejki oczekujących na zgodę i zwiększa licznik zgód o $ 1 $
            \end{enumerate}
        \end{enumerate}
    \item $ w $ wybiera losowo liczbę jednostek wina $ v $, i rozsyła ją do wszystkich studentów wiadomością \texttt{WINEMAKER\_BROADCAST}
    \item $ w $ oczekuje na wiadomość $ m $ typu \texttt{STUDENT\_BROADCAST} z indeksem bezpiecznego miejsca równym $ b $
    \item $ w $ rozyła wiadomość \texttt{WINEMAKER\_ACKNOWLEDGE} do winiarzy w kolejce oczekujących na zgodę
    \item Wróć do \ref{alg:wmk:start}
\end{enumerate}



\end{document}